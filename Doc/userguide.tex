\documentclass[12pt,a4paper]{article}
 \def\version{4.1}

 \usepackage{epsfig}
 \usepackage{html}
  %\def\htmladdnormallink#1#2{#1}

\begin{document} 
\author{}
\date{}
\title{
 \epsfig{figure=quantum_espresso,width=5cm}\hskip 2cm
 \epsfig{figure=democritos,width=6cm}\vskip 2cm
  % title
  \huge User's Guide for Quantum ESPRESSO \smallskip
  \Large (version \version)
}
\maketitle

\tableofcontents

\section{Introduction}

This guide covers the installation and usage of Quantum ESPRESSO
(opEn-Source Package for Research in Electronic Structure, Simulation,
and Optimization), version \version.

The Quantum ESPRESSO distribution contains the following core packages 
for the calculation of electronic-structure properties within
Density-Functional Theory, using a Plane-Wave basis set and pseudopotentials:
\begin{itemize}
  \item PWscf (Plane-Wave Self-Consistent Field),
  \item CP (Car-Parrinello).
\end{itemize}
It also includes the following more specialized packages:
\begin{itemize}
  \item PHonon:
        phonons with Density-Functional Perturbation Theory,
  \item PostProc: various utilities for data prostprocessing,
  \item PWcond:
        ballistic conductance,
  \item GIPAW  (Gauge-Independent Projector Augmented Waves):
        EPR g-tensor and NMR chemical shifts,
  \item XSPECTRA:
        K-edge X-ray adsorption spectra,
  \item vdW:
        (experimental) dynamic polarizability, 
  \item Wannier90:
        maximally localized Wannier functions.
\end{itemize}
Finally, the following auxiliary codes are included:
\begin{itemize}
\item PWgui (Graphical User Interface for PWscf): 
      a graphical interface for producing input data files for PWscf,
\item atomic: 
      a program for atomic calculations and generation of pseudopotentials,
\item iotk:
      an Input-Output ToolKit.
\end{itemize}
This guide documents PWscf, CP, PHonon, PostProc, PWcond. 
The remaining packages have separate documentation.

The Quantum ESPRESSO codes work on many different types of 
Unix machines,
including parallel machines using Message Passing Interface (MPI).
Running Quantum ESPRESSO on Mac OS X and MS-Windows is also possible: 
see section \ref{installation}, ``Installation''.

Further documentation, beyond what is provided in this guide, can be found in:
\begin{itemize}
  \item the Quantum ESPRESSO Wiki\\
   (http://www.quantum-espresso.org/wiki/index.php/Main\_Page) ;
  \item the Doc/ directory of the Quantum ESPRESSO distribution,
   containing a detailed description of all input data for all codes
   in the INPUT\_* files (in .txt, .html, .pdf formats);
\item the pw\_forum mailing list (pw\_forum@pwscf.org).
   You can subscribe to this list and browse and search its archives 
   from the Quantum ESPRESSO web site 
   (http://www.quantum-espresso.org/tools.php).\\
   Only subscribed users can post. Please search the archives 
   before posting: your question may have already been answered.
\end{itemize}

This guide does not explain solid state physics and its computational methods.
If you want to learn that, you should read a good textbook, such as e.g.
the book by Richard Martin:
{\em Electronic Structure: Basic Theory and Practical Methods},
Cambridge University Press (2004).

\subsection{What can Quantum ESPRESSO do}

PWscf can currently perform the following kinds of calculations:
\begin{itemize}
  \item ground-state energy and one-electron (Kohn-Sham) orbitals
  \item atomic forces, stresses, and structural optimization
  \item molecular dynamics on the ground-state Born-Oppenheimer surface, 
   also with variable cell
  \item Nudged Elastic Band (NEB) and Fourier String Method Dynamics (SMD)
  for energy barriers and reaction paths
  \item macroscopic polarization and finite electric fields via 
  the modern theory of polarization (Berry Phases)
\end{itemize}
All of the above works for both insulators and metals, 
in any crystal structure, for many exchange-correlation functionals
(including spin polarization, DFT+U, exact exchange), for
norm-conserving (Hamann-Schluter-Chiang) pseudopotentials in 
separable form or Ultrasoft (Vanderbilt) pseudopotentials 
or Projector Augmented Waves (PAW) method.
Non-collinear magnetism and spin-orbit interactions 
are also implemented.  
Finite electric fields are implemented also using a supercell approach.

PHonon can perform the following types of calculations:
\begin{itemize}
  \item phonon frequencies and eigenvectors at a generic wave vector,
  using Density-Functional Perturbation Theory
  \item effective charges and dielectric tensors
  \item electron-phonon interaction coefficients for metals
  \item interatomic force constants in real space
  \item third-order anharmonic phonon lifetimes
  \item Infrared and Raman (nonresonant) cross section
\end{itemize}
PHonon can be used whenever PWscf can be used, with the exceptions of
DFT+U and exact exchange. PAW is not implemented for higher-order 
response calculations.

The package QHA (Quasi-Harmonic Approximation) for vibrational
free energy calculations
was contributed by Eyvaz Isaev (Moscow Steel and Alloy Inst and
Linkoping and Uppsala Univ.)

PostProc can perform the following types of calculations:
\begin{itemize}
  \item Scanning Tunneling Microscopy (STM) images;
  \item plots of Electron Localization Functions (ELF);
  \item Density of States (DOS) and Projected DOS (PDOS);
  \item L\"owdin charges;
  \item planar and spherical averages;
\end{itemize}
plus interfacing with a number of graphical utilities and with 
external codes.

\subsection{People}

In the following, the cited affiliation is the one where the last known 
contribution was done and may no longer be valid.

The maintenance and further development of the Quantum ESPRESSO code
is promoted by the DEMOCRITOS National Simulation Center 
of CNR-INFM
(Italian Institute for Condensed Matter Physics) under the 
coordination of
Paolo Giannozzi (Univ.Udine, Italy), with the strong support
of the CINECA National Supercomputing Center in Bologna under 
the responsibility of Carlo Cavazzoni.
     
The PWscf package (originally including PHonon and PostProc)
was originally developed by Stefano Baroni, Stefano
de Gironcoli, Andrea Dal Corso (SISSA), Paolo Giannozzi, and many others.
We quote in particular:
\begin{itemize}
  \item Matteo Cococcioni (MIT) for DFT+U implementation.
  \item David Vanderbilt's group at Rutgers for Berry's phase calculations.
  \item Michele Lazzeri (Paris VI) for the 2n+1 code and Raman cross section
 calculation with 2nd-order response.
  \item Ralph Gebauer (ICTP, Trieste) and Adriano Mosca Conte (SISSA, Trieste) 
for noncolinear magnetism.
  \item Carlo Sbraccia (Princeton) for NEB, Strings method, Metadynamics, for 
improvements to structural optimization and to many other parts of the code.
  \item Paolo Umari (Democritos) for finite electric fields.
  \item Renata Wentzcovitch (Univ.Minnesota) for variable-cell molecular 
dynamics.
  \item Lorenzo Paulatto (SISSA) for PAW implementation, built upon previous 
work by Guido Fratesi (Univ.Milano Bicocca) and Riccardo Mazzarello (SISSA).
\item Filippo Spiga (Univ. Milano Bicocca) for mixed SMP-OpenMP
 parallelization.
 \item Ismaila Dabo (INRIA, Palaiseau) for electrostatics with
 free boundary conditions.
\end{itemize}

The CP code is based on the original code written by Roberto Car and
Michele Parrinello. CP was developed by Alfredo Pasquarello (IRRMA, Lausanne),
Kari Laasonen (Oulu), Andrea Trave, Roberto Car (Princeton), 
Nicola Marzari (MIT), Paolo Giannozzi, and others.
FPMD, later merged with CP, was developed by Carlo Cavazzoni, 
Gerardo Ballabio (CINECA), Sandro Scandolo (ICTP), 
Guido Chiarotti (SISSA), Paolo Focher, and others.
We quote in particular:
\begin{itemize}
  \item Carlo Sbraccia (Princeton) for NEB and Metadynamics.
  \item Manu Sharma (Princeton) and Yudong Wu (Princeton) for maximally 
localized Wannier functions and dynamics with Wannier functions.
  \item Paolo Umari (MIT) for finite electric fields and conjugate gradients.
  \item Paolo Umari and Ismaila Dabo for ensemble-DFT.
  \item Xiaofei Wang (Princeton) for META-GGA.
  \item The Autopilot feature was implemented by Targacept, Inc.
\end{itemize}

PWcond was written by Andrea Dal Corso and
Alexander Smogunov (SISSA)

GIPAW (http://www.gipaw.org) was written by Davide Ceresoli
(MIT), Ari Seitsonen, Uwe Gerstmann,  Francesco Mauri (Univ.
Paris VI) 

PWgui was written by Anton Kokalj (IJS Ljubljana) and is based on his
GUIB concept (http://www-k3.ijs.si/kokalj/guib/).

atomic was written by Andrea Dal Corso and it is the result 
of many additions to the original code by Paolo Giannozzi 
and others. Lorenzo Paulatto wrote the extension to PAW.

The input/output toolkit ''iotk'' (http://www.s3.infm.it/iotk) used in 
Quantum Espresso was written by Giovanni Bussi (ETHZ and S3 Modena).

Wannier90 (http://www.wannier.org/) was written by A. Mostofi, J. Yates, 
Y.-S Lee (MIT).

XSPECTRA was written by Matteo Calandra (Univ. Paris VI).

Other relevant contributions to Quantum Espresso:
\begin{itemize}
  \item Gerardo Ballabio wrote the first "configure" for Quantum Espresso.
  \item The calculation of the finite (imaginary) frequency molecular
polarizability using the approximated Thomas-Fermi  + von Weizaecker
scheme (VdW) was contributed by Huy-Viet Nguyen (SISSA).
  \item The calculation of RPA frequency-dependent complex dielectric
function was contributed by Andrea Benassi (S3 Modena).
  \item The initial BlueGene porting was done by Costas Bekas and
Alessandro Curioni (IBM Zurich).
  \item Simon Binnie (Univ.College London), Davide Ceresoli,
Andrea Ferretti (S3), Guido Fratesi, Axel Kohlmeyer (UPenn),
Konstantin Kudin (Princeton), Sergey Lisenkov (Univ.Arkansas), 
Nicolas Mounet (MIT), Guido Roma (CEA), Pascal Thibaudeau (CEA), 
answered questions on the mailing list, found bugs, helped in 
porting to new architectures, wrote some code.
\end{itemize}

An alphabetical list of further contributors includes: Dario Alf\`e, 
Alain Allouche, 
Francesco Antoniella, Mauro Boero, Nicola Bonini, Claudia Bungaro, 
Paolo Cazzato, Gabriele Cipriani, Jiayu Dai, Cesar Da Silva, 
Alberto Debernardi, Gernot Deinzer, 
Martin Hilgeman,  Yosuke Kanai, Nicolas Lacorne, Stephane Lefranc,
Kurt Maeder, Andrea Marini, 
Pasquale Pavone,  Mickael Profeta, Kurt Stokbro, Paul Tangney, 
Antonio Tilocca, Jaro Tobik, 
Malgorzata Wierzbowska, Silviu Zilberman, 
and let us apologize to everybody we have forgotten.
    
This guide was mostly written by Paolo Giannozzi.
Gerardo Ballabio and Carlo Cavazzoni wrote the section on CP..

\subsection{Contacts}

The web site for Quantum ESPRESSO is http://www.quantum-espresso.org/.
Releases and patches of Quantum ESPRESSO can be downloaded from this
site or following the links contained in it. The main entry point for 
developers is the QE-forge web site: http://www.qe-forge.org/.

Announcements about new versions of Quantum ESPRESSO are available 
via a low-traffic mailing list Pw\_users: (pw\_users@pwscf.org). You can
subscribe (but not post) to this list from the Quantum ESPRESSO web site.
    
The recommended place where to ask questions about installation and
usage of Quantum ESPRESSO, and to report bugs, is the Pw\_forum mailing
list (pw\_forum@pwscf.org). Here you can obtain help from the developers
and many knowledgeable users. You can browse and search its archive from 
the Quantum ESPRESSO web site, but you have to subscribe in order to post 
to the list.
Please search the archives before posting: your question may have already
been answered.
{\bf Important notice:} only messages that appear to come from the 
registered user's e-mail address, in its {\em exact form}, will be
accepted. Messages "waiting for moderator approval" are automatically 
deleted with no further processing (sorry, too much spam). In case of 
trouble, carefully check that your return e-mail is the correct one 
(i.e. the one you used to subscribe).

The Pw\_forum mailing list is also the recommanded place where to 
contact the developers of Quantum ESPRESSO.
 
\subsection{Terms of use}

Quantum ESPRESSO is free software, released under the 
GNU General Public License 
(http://www.pwscf.org/License.txt, or the file License in the
distribution).
    
All trademarks mentioned in this guide belong to their respective owners.
    
We shall greatly appreciate if scientific work done using this code will 
contain an explicit acknowledgment and the following reference:
\begin{quote}
P. Giannozzi et al., {\bf TO BE UPDATED}
\end{quote}
Note the form {\sc Quantum ESPRESSO} for textual citations of the code.
Pseudopotentials should be cited as )(for instance)

[PSEUDO] We used the pseudopotentials C.pbe.rrjkus.UPF
and O.pbe.vbc.UPF from the http://www.quantum-espresso.org 
distribution.

\end{document}
